% !TeX root = ../main.tex

% 英文誌謝
\begin{acknowledgement*}

I often eat at foreign friends' houses. When the candles are lit, the dishes are ready, and the host is in place, the host’s little boy or girl always raises his little hand, bows his head to thank God for the gift, and welcomes the guest.

When I first arrived in the United States, I was often embarrassed. Because of the habit developed in the country, I started before sitting down.

In the future, when you go to a friend's house for dinner, always ask yourself first, don't forget today, don't start too fast! In the past few years, I have become very used to it. But I always think that it is just a different kind of custom ceremony. In my opinion, forgetting or not forgetting does not matter much.

Once the year before last, I went to eat at the same house again. But this time, the host’s grandmother thanked him for the meal. Her snow-white hair, trembling voice, under the flickering candlelight, reminded me of my childhood grandmother. That night, I suddenly felt that my calm and watery emotions turned up against the huge waves.

When I was young, every winter night, our big family gathered around a big round table to eat. I always sit next to my grandmother, who always touches my head and says, "God rewards our family for a full meal. Remember, no grain of rice is allowed in the rice bowl. If the food is spoiled, God will not give us food. Up."

When I just started elementary school, my righteous thoughts brought down idols and dispelled superstitions. My school was the former Guandi Temple, and my desk was the offering table. I once drew glasses for Zhou Cang and a beard for Guan Ping. My grandmother's words, God, I think it is redundant and outdated.

However, I respect my grandparents very much, because they really earned the food, and they did set up this family.

I thank the grandparents in front of me, and I don't have to thank the vague God.

This idea has not changed with age. How many years have passed in this philosophy.

After dinner in this foreign family, because of this old foreign lady, I think of my childhood; because of my childhood, I think of a series of very strange phenomena.

Grandfather gritted his teeth in the wind and rain every year, and grandmother gritted his teeth in the tea and rice every year. They knew that they had to drip the sweat from their eyebrows to pick up the wheat ears in the field. Why did they thank God? I am obviously a kid, eating and playing, but why don't I thank God?

This strange state of mind has always been a mystery in my heart.

Until the year before last, when I was in Princeton, I browsed Einstein's "The World I Seen" and got new insights.

This is a non-scientific collection of essays, specifically containing some of Einstein's speeches at the memorial, the welcome party, and the funeral of a friend.

When I was reading this book, I suddenly discovered that Einstein wanted to give the audience an impression as much as possible: that is, his contribution was either due to A or B, and was not very relevant to Einstein himself.

Even the new and original special theory of relativity that has been created since ancient times has no reference to quote, but it flew in the last sky, "Thanks to my colleague and friend Besso for the discussion of the phase."

Other articles, such as the general theory of relativity, which has been struggling for more than ten years, are partly promoted to the cooperation of friends in the past; this kind of humility, this kind of no credit is rare in the history of science.

I just thought, I can't live so much, why? Like Einstein to the theory of relativity, like my grandmother to my family.

After several years of running around, doing some research, writing a few academic articles, and really making some small contributions, I have a new consciousness: that is, no matter what, too many people get it, and it’s out of it. There are too few people who own it.

Because there are too many people to thank, just thank God. No matter what, it doesn't need the love and inheritance of the ancestors, it needs the support and cooperation of everyone, and also wait for the opportunity to come. The more I have done something, the more I feel that my contribution is small.

As a result, entrepreneurs will naturally think of God, but the prodigal will always think of themselves.

The introduction of the introduction does not say the price, the price is also the same. This is the most perfect story composed of one of our most perfect personality in China. Why didn't Jie Zhitui say anything, because he felt that the merits of greed for the heavens were his own power, which a gentleman would disdain and should not do.

When Einstein first arrived in Princeton, the director discussed with him the issue of remuneration. He said five thousand. The director said, "I give you five thousand, how can I give a college graduate? It's still fifteen thousand yuan!" Isn't this a foreign recommendation?

Why did Jie Zhitui and Einstein specialize in such stupid things? Make great contributions, but don't take credit for it. They know that for doing things and meritorious service, there are more people who can cooperate with others, and fewer people who can do things by themselves. So it is natural to have a feeling of thanking everyone and God.

Let us look back and think about it, how much stronger China was in the past 50 or 60 years than when I was 7 or 8 years old! If historians write the history of our country in the past fifty or sixty years, they must name it the arrogant and naive, lawless and lawless era.

No matter which line or sector, most of them are boasting and deceiving themselves. The days are long, and even I believe it is true, and the catastrophe is terrible.

Because you haven't done any real things, haven't built any real gong, naturally you won't feel thankful.

Philosophers know that this symptom is the most terrifying, so they have created many characters and stories that know good and bad.

A person asked a writer, I remember it was Hugo, "If all the books in the world need to be burned and only one is allowed, what should be kept?" Hugo said without hesitation, "Just keep< "The Book of Job"." Job is the introduction in the "Bible". The rich also thank the heavens, the poor also thank the heavens, the sickness also thank the heavens, and the sufferings also thank the heavens.

Our ideological world is still in the chaotic and naive period, and needs Job's spirit and the enlightenment. This enlightenment is: one porridge, one meal, half a thread, half a thread, it is the crystallization of the blood and sweat of how many years and how many people. Thank you for being thankful.

\end{acknowledgement*}

% 中文誌謝
\begin{acknowledgement}

常到外國朋友家吃飯。當蠟燭燃起,菜餚布好,客主就位,總是主人家的小男孩或小女孩舉起小手,低頭感謝上天的賜予,並歡迎客人的到來。

我剛一到美時,常鬧得尷尬。因為在國內養成的習慣,還沒有坐好,就開動了。

以後凡到朋友家吃飯時,總是先囑咐自己,今天不要忘了,可別太快開動啊!幾年來,我已變得很習慣了。但我一直認為只是一種不同的風俗儀式,在我這方面看來,忘或不忘,也沒有太大的關係。

前年有一次,我又是到一家去吃飯。而這次卻是由主人家的祖母謝飯。她雪白的頭髮,顫抖的聲音,在搖曳的燭光下,使我想起兒時的祖母。那天晚上,我忽然覺得我平靜如水的情感翻起滔天巨浪來。

在小時候,每當冬夜,我們一大家人圍域個大圓桌吃飯。我總是坐在祖母身旁,祖母總是摸著我的頭說;「老天爺賞我們家飽飯吃,記住,飯碗裏一粒米都不許剩,要是糟蹋糧食,老天爺就不給咱們飯了。」

剛上小學的我,正念打倒偶像,破除迷信,我的學校就是從前的關帝廟,我的書桌就是供桌。我曾給周倉畫上眼鏡,給關平戴上鬍子,祖母的話,老天爺也者,我覺得是既多餘,又落伍的。

不過,我卻很尊敬我的祖父母,因為這飯確實是他們掙的,這家確實是他們立的。

我感謝面前的祖父母,不必感謝渺茫的老天爺。

這種想法並未因年紀長大而有任何改變。多少年,就在這種哲學中過去了。

我在這個外國家庭晚飯後,由於這位外國老太太,我想起我的兒時;由於我的兒時,我想起一串很奇怪的現象。

祖父每年在「風裏雨裏的咬牙」,祖母每年在「茶裏飯裏的自苦」,他們明明知道要滴下眉毛上的汗珠,才能撿起田中的麥穗,而為什麼要謝天?我明明是個小孩子,混吃混玩,而我為什麼卻不感謝老天爺?

這種奇怪的心理狀態,一直是我心中的一個謎。

一直到前年,我在普林斯頓,瀏覽愛因斯坦的《我所看見的世界》,得到了新的領悟。

這是一本非科學性的文集,專載些愛因斯坦在紀念會上啦、在歡迎會上啦、在朋友的葬禮中,他所發表的談話。

我在讀這本書時忽然發現愛因斯坦想盡量給聽眾一個印象:即他的貢獻不是源於甲,就是由於乙,而與愛因斯坦本人不太相干似的。

就連那篇亙古以來嶄新獨創的狹義相對論,並無參考可引,卻在最後天外飛來一筆,「感謝同事朋友貝索的時相討論。」

其他的文章,比如奮鬥苦思了十幾年的廣義相對論,數學部分推給了昔年好友的合作;這種謙抑,這種不居功,科學史中是少見的。

我就想,如此大功而竟不居,為什麼?像愛因斯坦之於相對論,像我祖母之於我家。

幾年來自己的奔波,作了一些研究,寫了幾篇學術文章,真正做了一些小貢獻以後,才有了一種新的覺悟:即是無論什麼事,得之於人者太多,出之於己者太少。

因為需要感謝的人太多了,就感謝天罷。無論什麼事,不是需要先人的遺愛與遺產,即是需要眾人的支持與合作,還要等候機會的到來。越是真正做過一點事,越是感覺自己的貢獻之渺小。

於是,創業的人,都會自然而然的想到上天,而敗家的人卻無時不想到自己。

介之推不言祿,祿亦弗及。這是我們中國的一個最完美的人格所構成的一個最完美的故事。介之推為什麼不言祿,因為他覺得貪天之功以為己力,是君子所不屑為,也是君子所不應為的。

愛因斯坦剛到普林斯頓時,主任與他商量報酬問題,他說五千。主任說:「給你五千,如何給一個大學畢業生呢?還是算一萬五千元罷!」這不是外國的介之推嗎?

為什麼介之推與愛因斯坦專幹這類傻事?立過大功,而不居功若此。他們知道作事與立功,得之於眾人合作者多,得之於自己逞能者少。於是很自然的產生一種感謝眾人、感謝上天的感覺。

我們回頭想一想,五六十年來的中國比我七八歲時的思想能強幾何!史家如果寫這五六十年來的我國歷史時,一定命名為狂妄而幼稚,無法與無天的時代。

無論哪一行、哪一界,多是自吹自擂,自欺自騙。日子長了,連自己也信以為真了,而大禍至矣。

因為沒有做任何真正的事,沒有建任何真正的功,自然而然不會有謝天的感覺。

哲學家們知道這個症候最為可怕,所以造出許多知好知歹的人物與故事來。

有一個人問一位文學家,我記得是雨果罷,「如果世界上的書全需要燒掉,而只許留一本,應留什麼?」雨果毫不猶豫的說:「只留〈約伯記〉。」約伯是《聖經》裏面的介之推,富亦謝天,貧亦謝天,病亦謝天,苦亦謝天。

我們的思想界尚在混沌幼稚時期,需要約伯的精神,需要介之推的覺悟。這個覺悟即是:一粥一飯,半絲半縷,都是多少年、多少人的血汗結晶。感謝之情,無由表達,還是謝天罷。

\end{acknowledgement}