% !TeX root = ../main.tex

\begin{abstract*}

The abstract is an important component of your thesis. Presented at the beginning of the thesis, it is likely the first substantive description of your work read by an external examiner. You should view it as an opportunity to set accurate expectations.
The abstract is a summary of the whole thesis. It presents all the major elements of your work in a highly condensed form.
An abstract often functions, together with the thesis title, as a stand-alone text. Abstracts appear, absent the full text of the thesis, in bibliographic indexes such as PsycInfo. They may also be presented in announcements of the thesis examination. Most readers who encounter your abstract in a bibliographic database or receive an email announcing your research presentation will never retrieve the full text or attend the presentation.
An abstract is not merely an introduction in the sense of a preface, preamble, or advance organizer that prepares the reader for the thesis. In addition to that function, it must be capable of substituting for the whole thesis when there is insufficient time and space for the full text.

Currently, the maximum sizes for abstracts submitted to Canada's National Archive are 150 words (Masters thesis) and 350 words (Doctoral dissertation).
To preserve visual coherence, you may wish to limit the abstract for your doctoral dissertation to one double-spaced page, about 280 words.
The structure of the abstract should mirror the structure of the whole thesis, and should represent all its major elements.
For example, if your thesis has five chapters (introduction, literature review, methodology, results, conclusion), there should be one or more sentences assigned to summarize each chapter.

As in the thesis itself, your research questions are critical in ensuring that the abstract is coherent and logically structured. They form the skeleton to which other elements adhere.
They should be presented near the beginning of the abstract.
There is only room for one to three questions. If there are more than three major research questions in your thesis, you should consider restructuring them by reducing some to subsidiary status.

The most common error in abstracts is failure to present results.
The primary function of your thesis (and by extension your abstract) is not to tell readers what you did, it is to tell them what you discovered. Other information, such as the account of your research methods, is needed mainly to back the claims you make about your results.
Approximately the last half of the abstract should be dedicated to summarizing and interpreting your results.

\end{abstract*}


\begin{abstract}

一個好的摘要應該具有以下五個特點。
APA提到一篇好的摘要需有:
「精確」(accurate) :摘要應該是文章的精簡版,所以它的內容不應該超過文章內容的範圍。
「完備」(self-contained) :摘要是一篇獨立的文章,所以一些可能讓讀者讀不懂的東西不要放在摘要中;如果一定要放一些讀者讀完摘要還可能不懂的東西時,作者應該在摘要中加以解釋。例如,如果摘要中含有一些冷僻、不常見的專有名詞,則作者應該在摘要中說明名詞的定義,這樣才不會讓讀者感到生澀難懂。
「簡潔明確」(concise and specific) :字數用詞切中重點。
「非評論性」(non-evaluative) :不要在摘要中評價研究的發現。
「連貫性」及「易讀性」(coherent and readable):文章寫得很淺顯、流暢、易懂,科學報導最重要的目的是傳達信息,作者應該盡量的用常見的字彙、合乎中英文文法的清晰筆調撰寫文章。

摘要寫作的步驟:
1. 仔細閱讀原文。
尤其是論文中的主要部分:研究目的或動機、研究方法、研究範圍、研究結果、結論等部分。
段落的標題或圖表可作為摘要寫作時的引導。
2. 把原文放一邊,開始寫作初稿。
不要只是照抄原文的字句,這樣反而容易錯失真正的重點。
儘可能在不扭曲原意的情況下,將要濃縮的重點「換句話說」。
因為字數有限,所以要長話短說、言簡意賅,避免不必要的語助詞及贅字。
表達時應避免使用過多的術語,造成閱讀時的困難。語句應清楚精確,避免模擬兩可。
3. 校訂初稿。
調整摘要架構,以利閱讀。
去蕪存菁,檢查原文重點是否有遺漏或不全之處,切勿加入個人評論。
注意上下文之連結,務使文章能一氣呵成。
檢查語法、用字遣詞及標點符號是否正確。
將文章大聲地多唸幾次,或請同學批評指教,找出文章之盲點,並加以修正。

\end{abstract}